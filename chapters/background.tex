\chapter{Background}

This chapter concludes the fundamental knowledge that is needed to comprehend the upcoming practical work. Firstly, an introduction to cloud computing will be held. Next, a throrough understanding of honeypots is given. Lastly, we introduce some concepts of intrusion detection systens.

\section{Cloud Computing}

Nowadays it is one of the well-known keywords and has been used by vary large companies such as Google, or Amazon, however, the term ''cloud computing'' dates back to the late 1996, when a small group of technology executives of Compaq Computer framed new business ideas around the Internet.\cite{regalado2020} In this section, we want to give basic unterstandings of cloud computing, and give a short introduction to HeiCloud.

\subsection{Definition of Cloud Computing}

Considering the definition of Brian Hayes, cloud computing is ''a shift in the geography of computation'' \cite{hayes2008}. Thus, computational workload is moved away from local instances towards services and data centers that provide the need of users \cite{Armbrust2010}.

Considering the definition of the \acrfull{nist}, cloud computing ''is a model for enabling ubiquitous, convenient, on-demand network access to a shared pool of configurable computing resources (e.g., networks, servers, storage, applications, and services) that can be rapidly provisioned and released with minimal management effort or service provider interaction.'' \acrshort{nist} the geographical shift, but also mentions the . Morever, the term is composed of five essential characteristics, three service models (see \ref{subsec:cloud-service}), and four deployment models (see \ref{subsec:cloud-deployment}.)

\textit{On-demand-self-service}

\textit{Broad network access}

\textit{Resource pooling}

\textit{Rapid elasticity}

\textit{Measured service}

\subsection{Service models}
\label{subsec:cloud-service}

\acrfull{saas}

\acrfull{paas}

\acrfull{iaas}

\subsection{Deployment models}
\label{subsec:cloud-deployment}

Private Cloud

Community Cloud

Public Cloud

Hybrid Cloud

\subsection{Cloud Security}

\cite{Nithin2012}

\subsection{HeiCloud}

\section{Honeypots}

The first public honeypot \cite{Spitzner2003}

\subsection{Definition of a Honeypot}

On the Internet there are a dozen of defintions for honeypots. Thus, to cope with all the subtle differences, we want to take a closer look at some of the definitions and narrow down our own one.

Spitzner defines honeypots as a ''security resource whose value lies in being probed, attacked, or compromised.''\cite{Spitzner2003}

High-interaction honeypots\\

Low-interaction honeypots\\

Pure honeypots\\

%\subsection{Honeyd}

%\subsection{Configuration Honeyd}

\subsection{Honeynets}

\cite{Spitzner2003}

\subsection{Legal Issues}

\cite{Spitzner2003}

\section{Intrusion Detection System}
