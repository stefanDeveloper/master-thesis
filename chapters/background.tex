\chapter{Background}

This chapter concludes the fundamental knowledge that is needed to comprehend the upcoming practical work. Firstly, an introduction to cloud computing will be held. Next, a throrough understanding of honeypots is given. Lastly, we introduce some concepts of intrusion detection systens.

\section{Cloud Computing}

Nowadays it is one of the well-known keywords and has been used by vary large companies such as Google, or Amazon, however, the term ''cloud computing'' dates back to the late 1996, when a small group of technology executives of Compaq Computer framed new business ideas around the Internet.\cite{regalado2020} In this section, we want to give basic unterstandings of cloud computing, and give a short introduction to HeiCloud.

\subsection{Definition of Cloud Computing}

Considering the definition of Brian Hayes, cloud computing is ''a shift in the geography of computation'' \cite{hayes2008}. Thus, 

\subsection{Service models}

Software-as-a-Service\\

Platform-as-a-Service\\

Infrastructure-as-a-Service\\

\subsection{Deployment models}

Public Cloud\\

Private Cloud\\

Hybrid Cloud\\

\subsection{Cloud Security}

\cite{Nithin2012}

\subsection{HeiCloud}

\section{Honeypots}

The first public honeypot \cite{Spitzner2003}

\subsection{Definition of a Honeypot}

\cite{Spitzner2003}

High-interaction honeypots\\

Low-interaction honeypots\\

Pure honeypots\\

%\subsection{Honeyd}

%\subsection{Configuration Honeyd}

\subsection{Honeynets}

\cite{Spitzner2003}

\subsection{Legal Issues}

\cite{Spitzner2003}

\section{Intrusion Detection System}
