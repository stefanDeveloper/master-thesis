\chapter{Conclusion}

We have shown that cybercrime activities can be spotted by using honeypot solutions.
Thus, the initial question if honeypots contribute to a more secure infrastructure has been answered successfully.
We can confirm this assumption based on our results we have achieved in the cloud and in the university network.
Our first approach was to collect data by the help of the T-Pot solution and compare them to a previous study of similiar cloud providers.
It has shown that these activities increased significantly.
Considering the services that are excluded by default of the firewall, heiCLOUD has received more attacks than ever, putting it on the first place compared to other cloud providers.
We have seen various attacks such as in RDP, VoIP, and SSH.
Outstanding are the amount of cryptocurrency related attacks which reflects the current situation of highly traded GPUs in the market.
In addition, the latest attacks like the Apache vulnerability in version $2.49.0$ could be traced back to very early stages, showing how fast attackers adapt to new vulnerabilities.
Our assumption is that the vast majority of executed attacks on our instance originates from bots.

Next, we have focused at the internal network of the university and implemented a new concept to detect every single packet that have been sent to a host machine.
The MADCAT solution in conjunction with IDS tools helped us identify the open port 113 that has been used to deploy attacks.
We have shown that known attackers with an IP addresses originated from Russia probed our instance, and we assume further exploits would be proceeded.
In retrospect, this helped to remove the port from the permits in the firewall, thus, we have improved the security at the University Heidelberg.
Any other suspicious behavior in the eduroam network could not be registered, proving that the firewall works by design.

Moreover, we have shown that honeypots like Cowrie have a fundamental flaw because they rely on off-the-shelf libraries.
These libraries often re-implement protocol behavior, thus, adding a subtle difference to the response.
Based on a cosine similiartiy coefficient, adversaries could detect honeypots before deploying any attack, thus, 

Future work



