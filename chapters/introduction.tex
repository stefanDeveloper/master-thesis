\chapter{Introduction}

In this chapter an introduction about the work that is done in this thesis will be held. First, a problem is described to lay emphasis on the need. Next, we look at justifications, motivations, and benefits prior to our work. It strengthen the necessity of it. In addition, we set our research question that we try to answer later, and add limitations to it.

\section{Problem description}

\todo{Modern attacks, statistis}



Fast growing technology comes along with new security concerns. Especially in cloud computing due to access to large pools of data and computational resources, controlling access to services is becoming a tougher challenge nowadays. One known methodology to strengthen infrastructures is to learn from those who attacks it. Honeypots are a security resource whose value lies in being probed, attacked, or compromised.

\section{Justification, motivation and benefits}

However, controlling attackers in a cloud environment may differ from
other infrastructures.

\section{Research questions}

We want to answer the question, how can honeypots contribute to a safer cloud environment. This includes baiting adversaries to our honeypots, controlling their requests, and analyzing their attacks.

\section{Limitations}

Due to the fact that Heidelberg offers a cloud service, called
\enquote{HeiCloud}, we tailor our implementation to this service.