\chapter{Conclusion}

We have shown that cybercrime activities can be spotted by using honeypot solutions.
This successfully answered the original question of whether honeypots contribute to a more secure infrastructure.
We can confirm this assumption based on our results we have achieved in the cloud and in the university network.
Our first approach was to collect data by the help of the T-Pot solution and compare them to a previous study of similiar cloud providers.
It has shown that these activities increased significantly.
Considering the services that are excluded by default of the firewall, heiCLOUD has received more attacks than ever, putting it on the first place compared to other cloud providers.
We have seen various attacks such as in RDP, VoIP, and SSH.
Outstanding are the amount of cryptocurrency related attacks which reflects the current situation of highly traded GPUs on the market.
In addition, the latest attacks like the Apache vulnerability in version $2.49.0$ could be traced back to very early stages, showing how fast attackers adapt to new vulnerabilities.
Our assumption is that the vast majority of executed attacks on our instance originated from bots.

Next, we have focused at the internal network of the university and implemented a new concept to detect every single packet that have been sent to a host machine.
The MADCAT solution in conjunction with IDS tools helped us to identify the open port 113 that has been used to deploy attacks.
We have shown that known attackers with an IP address originating from Russia have probed our instance, and we assume further attacks would have been carried out.
In retrospect, this helped to remove the port from the permits in the firewall, thus, we have improved the security at the Heidelberg University.
Any other suspicious behavior in the eduroam network could not be registered, proving that the firewall works by design.

Moreover, we have shown that honeypots like Cowrie have a fundamental flaw because they rely on off-the-shelf libraries.
These libraries often re-implement protocol behaviors like OpenSSH, and adding a subtle difference to the response.
On the contrary, this deviation of responses can be used to detect honeypots on transport level.
Base on a cosine similiartiy coefficient, adversaries could spot honeypots before deploying any attack, thus, avoid exposures of newly developed attacks.
We have re-created the findings \citet{vetterl2020} claims in his work by adapting OpenSSH $8.8P1$, and testing it on different Debian instances.
Due to outdated algorithms, we have updated the key exchange initialization message to work with the latest version.
We have shown that the lastest Cowrie version $2.3.0$ results in a bad packet length because the local version string does not match the expected ones of the underlying library TwistedConch.
This result deviates fundamentally from OpenSSH.
Lastly, we tried to protect Cowrie from early exposure by hiding it in the background and tunneling requests through a customized OpenSSH daemon.
We have successfully fixed the generic weakness of Cowrie, so that, connecting to Cowrie works without running into a bad packet length error.
Our last chapter shows that honeypots are not flawless, and developers should be careful during the decision of additional libraries.

In conclusion, we have presented concepts to catch attackers for different scenarios, and shown that cybercrime activities have increased tremendously.
In addition, we have taken a deep dive into an edge-breaking study to detect honeypots on transport level, and have disguised Cowrie to block such activities.
Thus, using honeypots as a security parameter have been proven promising for further scientific research.
