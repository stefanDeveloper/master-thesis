\chapter{Conclusion}

This thesis has shown that organizations can spot malicious activities using honeypot solutions.
This result successfully answered the original question of whether honeypots contribute to a more secure infrastructure.
It can confirm this assumption based on its results in the cloud and in the university network.
The first approach was to collect data with the help of the T-Pot solution and compare them to a previous study of similar cloud providers.
It has shown that these activities increased significantly.
Considering the services excluded by default of the firewall, heiCLOUD has received more attacks than ever, putting it in the first place compared to other cloud providers.
It has seen various attacks in RDP, VoIP, and SSH.
Outstanding is the amount of cryptocurrency-related attacks that reflect the current situation of highly traded GPUs.
In addition, the latest attacks like the Apache vulnerability in version $2.49.0$ could be traced back to very early stages, showing how fast attackers adapt to new vulnerabilities.
The assumption is that most executed attacks on the instance originated from bots.

Next, this thesis has focused on the university's internal network and implemented a new concept to detect every single packet sent to a host machine.
The MADCAT solution, in conjunction with IDS tools, helped identify the open port 113 that has been used to deploy attacks.
It has shown that known attackers with an IP address originating from Russia have probed the instance, and as an assumption, further attacks would have been carried out.
In retrospect, this helped remove the port from the firewall's permits, thus improving the security at the Heidelberg University.
Any other suspicious behavior in the eduroam network could not be registered, proving that the firewall works as intented.

Moreover, honeypots like Cowrie have a fundamental flaw because they rely on off-the-shelf libraries.
These libraries often reimplement protocol behaviors like OpenSSH and add a subtle difference to the response.
On the contrary, this deviation of responses can be used to detect honeypots on the transport level.
Adversaries could spot honeypots before deploying any attack based on a cosine similarity coefficient, thus avoiding exposures to newly developed attacks.
The findings \citet{vetterl2020} claims in his work have been recreated by adapting OpenSSH $8.8P1$ and testing it on different Debian instances.
Due to outdated algorithms, the key exchange initialization message has been updated to work with the latest version.
It shows that the latest Cowrie version $2.3.0$ results in a bad packet length because the local version string does not match the expected ones of the underlying library TwistedConch.
This result deviates fundamentally from OpenSSH.
Lastly, an attempt to protect Cowrie from early exposure has been made by hiding it in the background and tunneling requests through a customized OpenSSH daemon.
This has successfully fixed the generic weakness of Cowrie so that connecting to Cowrie works without running into a bad packet length error.
The last chapter shows that honeypots are not flawless, and developers should be careful when deciding on additional libraries.

In conclusion, this thesis has presented concepts to catch attackers for different scenarios and shows that malicious activities have increased tremendously.
In addition, it has taken a deep dive into an edge-breaking study to detect honeypots on transport level and has disguised Cowrie to block such activities.
Thus, using honeypots as a security parameter has been proven promising for further scientific research.
