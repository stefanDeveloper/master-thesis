\thispagestyle{empty}
\begin{center}
    \begin{minipage}[c][0.48\textheight][b]{0.9\textwidth}
        \small
        \begin{center}
            \textbf{Abstract}
        \end{center}\par
        \vspace{\baselineskip}
        Nowadays, services which are accessible from the Internet face more attacks than ever.
        An attackers' objective is to exploit systems and use them for their own vicious purposes.
        Such efforts are on the rise as faulty systems can be discovered and compromised through Internet-wide scanning.
        Besides traditional security leverages, one known methodolgy is to learn from those who attacks it.
        A honeypot helps to gather information about an attacker.
        It is a security resource whose value lies in being probed, attacked, or compromised.
        Thus, it is an interesting question how honeypots can contribute to more secure infrastructure.
        In this work we will present a honeypot solution to investigate cybercrime activities in heiCLOUD, and show that attacks have increased significantly.
        We will try to catch attackers in restricted network zones at the Heidelberg University, and we discover leaks in the firewall allowing attackers to send malicious packets to the network.
        Moreover, we will consider an attacker's point of view by present a method to detect honeypots on transport level.
        Lastly, we mitigate these efforts with a customized OpenSSH server working as an intermediary instance.
    \end{minipage}\par
    \vfill
\end{center}