\chapter{Cloud Security with Honeypots}

\section{Introduction}

\todo{Daten sammeln mit T-Pot fuer Auswertungen, etc.. Bezug auf bisherige Arbeiten nehmen}

\section{HeiCloud}

\ac{iaas}\todo{Get information or paper about that}

\cite{heicloud2021}

\citet{Nithin2012}

\citet{Kelly2021}

\section{T-Pot}

\todo{Zusammenfassung tpot, funktionalitaet etc.}

\todo{Quelle jedes Honeypots hinzufügen}
\begin{itemize}
    \item adbhoney
    \item ciscoasa
    \item citrixhoneypot
    \item conpot
    \item cowrie
    \item ddospot
    \item dicompot
    \item dionaea
    \item elasticpot
    \item endlessh
    \item glutton
    \item heralding
    \item hellpot
    \item honeypy
    \item honeysap
    \item honeytrap
    \item ipphoney
    \item mailoney
    \item medpot
    \item rdpy
    \item redishoneypot
    \item snare
    \item tanner
\end{itemize}

\todo{übersicht tabelle}
\begin{table}[h]
    \centering
    \caption{}
    \begin{tabularx}{\linewidth}{l}
        \toprule
        \bottomrule
    \end{tabularx}
    \label{tab:overview-honeypots}
\end{table}

\todo{Übersicht zu Tpot, eigene Grafik}

\section{Data Analysis}

\begin{enumerate}
    \item Attack Profile
    \item Attack Source
    \item Attack Target
    \item Attack Frequency
    \item Attack Evolution
    \item Propagation of Attacks
    \item Attack Patterns
    \item Attack Root Cause Identification
    \item Attack Risk Assessment
    \item Exploit Detection
\end{enumerate}

\todo{übersicht tabelle}
\begin{table}[h]
    \centering
    \caption{}
    \begin{tabularx}{\linewidth}{l}
        \toprule
        \bottomrule
    \end{tabularx}
    \label{tab:overview-data-analysis}
\end{table}


\section{Results in HeiCloud}

\todo{show results}

\section{Summary}