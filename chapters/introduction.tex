\chapter{Introduction}

Recently, Europol\footnote{An agency that fights against terrorism, cybercrime, and other threats \cite{europol2021}} rose awareness of new cyber threats related to the pandamic.
As stated in their yearly \ac{iocta} report, scanning corporate infrastructures have been skyrocketing within the last 12 months by ransomware groups respectively increasing the malware usage.
Attackers use scans to find potential vulnerabilities in remote desktop sharing software, or \ac{vpn} in order to deploy malware and blackmail the company. \cite{iocta2020}
The rapid increase dates back to the pandamic and the correlated home office increase forcing companies to rapidly adapt their infrastructure.
Such changes come with the downside of adding new threats to the organization.
Latest findings...

Especially in cloud computing due to access to large pools of data and computational resources, controlling access to services is becoming a tougher challenge nowadays.
Besides traditional security leverages such as firewalls or intrusion detection systems, one known methodology to strengthen infrastructures is to learn from those who attacks it.
Honeypots are \enquote{a security resource whose value lies in being probed, attacked, or compromised} \cite{Spitzner2003}.
By collecting attacks, zero-day-exploits, or bots activities can be detected.
In retrospect, this would help to hardening infrastructures before proper damage occurs.
As a cloud provider, it is a crucial point if and how attacks on production servers could have been prevented.
Considering the Global Security Report by Trustwave, the amount of attacks doubled in 2019, and increased by $20\%$ in 2020 \cite{fahim2020}.
Respectively puttting cloud providers to the third most targeted environments for cyberattacks.
For instance, corporate and internal networks are the top of this ranking.
Heidelberg builds upon its own cloud provider.

As a conclusion, this thesis tries to answer the research question if honeypots contribute to a more secure infrastructure when using a cloud environment.
This includes 
On the contrary, we will present recent work to detect honeypots on transport level.


%However, this rise of attacks on cloud infrastructures has shown the importance why more security will be necessary in the future.
%Thus, making it a thrilling challenge for the upcoming years.

% TODO: Justification, motivation and benefits

% However, controlling attackers in a cloud environment may differ from other infrastructures.

% TODO:Research questions

%The following research questions have been answered in this thesis:

%\begin{enumerate}
%    \item How can honeypots contribute to a more secure cloud environment including baiting adversaries to our honeypots, and controlling their requests?
%    \item What is a preferable way to handle data management and visualization?
%    \item How can we analyze our data to get more information?
%\end{enumerate}

% TODO: Limitations

%Heidelberg offers a cloud service, called \enquote{HeiCloud}.
%Since  we tailor our implementation for this service.
%Moreover, it ecists a vast variety of different honeypots which bound us to a very few of them. 