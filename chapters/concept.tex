\chapter{Catching Attackers in Restricted Network Zones}
\label{chap:concept}

Our T-Pot identified a flood of threads when it was widely available on the Internet.
However, capacious networks do have separated compartments and usually services are not directly available without any protection.
Zoning is a well-known method to segment a network.
Heidelberg University applies zoning, and thus, it is an interesting question if any attacker probes services outside or within the network.
To detect any dubious packets in the network we present a concept using a honeypot-like detection tool.
We show that attacks in a restricted network zone of the Heidelberg University's internal network occur, and contributed to an adaption of the stateless firewall.
Thus, improving the security of the network.

\section{University Network}

Honeypots that are accessible via the Internet do receive a broad range of attacks.
As \citet{Spitzner2003} noted, a honeypot is not strictly bound to run in a \ac{dmz} or in a network with direct Internet access.
The correct location has to be chosen based on the goals of the honeypot.
For example, one goal could be to catch attackers behind a perimeter firewall to reveal leaks or vulnerabilities.
Beforehand, our honeypot was broadly available on the Internet, and attackers could probe it easily.
It collected on average $\numprint{55000}$ attacks per day, resulting in a total amount of $\numprint{786564}$ attacks.
Zoning a network into logical groups mitigates the risk of an open network.
Thus, our T-Pot would receive significantly fewer attacks in a controlled network zone.
A network infrastructure is segmented into the same communication security policies and security requirements.
For example, the Canadian government created their own baseline for infrastructures, called Baseline Security Architecture Requirements for Network Security Zones in the Government of Canada (ITSG-22).
The four most common zones are:
\begin{enumerate*}[label=(\roman*)]
    \item Public Zone (PZ) which is entirely open,
    \item Public Access Zone (PAZ) which interacts as an interface between the PZ and internal services,
    \item Operations Zone (OZ) which processes sensitive information, and
    \item Restricted Zone (RZ) including business critical services
\end{enumerate*}.
A network zone restricts access and controls data communication flows. \cite{csec2021}

\begin{figure}[ht]
    \centering
    \includegraphics[width=\textwidth]{figures/university-network.pdf}
    \caption[Draft of the University network]{Draft of the University network. The URZ firewall represents the stateful firewall. The \enquote{TagungsLAN} i}
    \label{fig:university-network}
\end{figure}

The network at the Heidelberg University includes a central stateless firewall (\ac{acl}) that enfolds all institutes.
It entails a default blacklist that blocks certain services such as SMTP, NCP, or SNMP, and a stateless filter provided by BelWÜ.
Inside the network, each institute has the possible to either use a pre-defined stateless firewall provided by the University Computing Center Heidelberg, or use a self-administrated firewall.
\autoref{fig:university-network} outlines the association between these components.
The internal \enquote{HDnet} enables the communication between institutes without leaving the internal network.
Institute firewalls can be set up by each institute and is self-administrated.
They do have the possibility to use \ac{soho} routers\footnote{\ac{soho} router is a broadband router used in small offices and home offices environments.} to disconnect certain network zones from the network.
It is recommended to configure the global \ac{acl} as a fall back solution in case of any downtime.
The University Computing Center Heidelberg offers stateless firewall for router interfaces or VLANs.
This stateful firewall whitelists certain services and splits up into four stages.
Each stage can be individually activated per router interface.
Its key value is to maintain a baseline security to avoid any misconfigurations and port scans.
\autoref{tab:overview-security-zone} outlines these stages including the IP address range.
Before applying one of these zones, the respective network has to oblige to client IP addresses below \ipAddress{129.206.218.240/24}.
In addition, \ipAddress{129.206.218.1} is allocated for the gateway.
A network has to adhere to these obligations if it applies any of these pre-defined stages.

\begin{table}
    \centering
    \caption[Overview of firewall stages]{
        Overview of firewall stages at the University Heidelberg.
        As an example we applied the rules to subnet \ipAddress{129.206.218.0/24}.
        Any rule does apply to any another subnet.
    }
    \begin{tabularx}{\linewidth}{l|XX}
        \toprule
        \textsc{Name} & \textsc{Description}                      & \textsc{range}                     \\
        \hline
        Stage 0       & Filters broadcast communication           & \ipAddress{129.206.218.0-15/24}    \\
                      & No filtering                              & \ipAddress{129.206.239.16-255/24}  \\
        \hline
        Stage 1       & Allows common network protocol            & \ipAddress{129.206.239.0-255/24}   \\
                      & Allows services                           & \ipAddress{129.206.239.240-255/24} \\
        \hline
        Stage 3       & Internet access only via internal proxies & \ipAddress{129.206.239.0-255/24}   \\
        \hline
        Stage 4       & Only internal network communication       & \ipAddress{129.206.239.0-255/24}   \\
        \bottomrule
    \end{tabularx}
    \label{tab:overview-security-zone}
\end{table}

An interesting question is if attackers have access to restricted zones at the Heidelberg University.
It arises during the research of T-Pot if an adversary would try to probe any hosts in the internal University network.
In order to detect such events we present honeypot-like packet detection application that helps to identify any threats in a network.
In addition, it offers to deploy multiple instance and collect their data in a centralized instance.

\section{Honeypot-like Connection Detection Tool}

Recording and investigating connection attempts assimilates new honeypots.
Respectively, we will present a new honeypot-like detection tool called MADCAT.
It has been developed by the BSI, and helps us to log any connection attempt being made on our host machine.
The acronym MADCAT stands for Mass Attack Detection Connection Acceptance Tools.
It works as a honeypot-like detection application with low interaction level.
Its key idea is to log every connection attempt and further process it to retrieve credentials, or shell exploitation.
\autoref{fig:madcat-architecture} gives an inside how MADCAT works.
It runs on an Ubuntu distribution either $18.04$ or $20.04$.
We have tested it on Ubuntu $18.04$.
It processes packets from any interface that has been configured.
As an example, we could process Ethernet and wireless packets.
MADCAT itself consists of six independent modules for TCP, UDP, ICMP, and RAW packets that communicating with each other through a pipeline.
A module is responsible for analyzing packets and logging the results in a queue.
In addition, UDP and TCP offers a proxy to tunnel packets to another service.
TCP postprocessor reads every 5 seconds the newly arrived TCP packets and processes them accordingly.
It resolves packets to log data including source IP address, protocol and event type.
The enrichment processor is the final process step.
Its purpose is to log all written packets of the queue in a specified format for further analyzation.
The key idea of MADCAT is to get an inside if attackers have access to a certain network.
In contrast to T-Pot, we do not want to know what specific attacks are operated on our honeypot.
Instead, we do want to ensure that no one else than authorized users have access.
Especially in high confidential areas, no attacker should be capable of sending even a single packet to a host in the network.
Tracking packets on a detailed level is not provided by the vast range of honeypots.

% TODO: Conecpt ueberarbeiten, beheben der Darstellung
\begin{figure}
    \centering
    \includegraphics[width=\textwidth]{figures/heicat-conecpt.pdf}
    \caption[Concept to detect connection attempts]{Concept to detect connection attempts. Kibana and Elasticsearch are deployed in heiCLOUD.}
    \label{fig:heicat-concept}
\end{figure}

In addition, we will deploy a T-Pot instance to have comparison data to our new concept.
We focus on the \ipAddress{129.206.218.0/24} and \ipAddress{147.142.0.0/16} subnet.
The \ipAddress{129.206.218.0/24} subnet is used within University Computing Center Heidelberg building.
Every client in the building has a compelling connection in this subnet.
Otherwise, an Internet connection would not be feasible.
The subnet \ipAddress{147.142.0.0/16} connects clients to \enquote{eduroam}\footnote{The eduroam is an international Wi-Fi internet access point for researchers.}.
Like the four stages of the institute firewall, the \enquote{eduroam} network, also called \enquote{Tagungslan}, builds various permits into the subnet.
One essential difference is that services like SMTP and HTTP are not allowed, so that, attackers are not able to deploy traps for users.
Moreover, each client is encapsulated in its own subnet that disables any communication to others clients.
Our instances are located in the building with IP addresses \ipAddress{129.206.219.65} and \ipAddress{129.206.219.88}.
\autoref{fig:heicat-concept} outlines our concept using MADCAT and a separate instance to visualize our data.
The first instance with IP address \ipAddress{129.206.5.157} provides Kibana, and Elasticsearch to visualize and crawl logs.
The honeypot with IP address \ipAddress{129.206.5.88} consists of MADCAT in conjunction with P0f, Suricata, and FATT.
Like T-Pot, we use Logstash to forward our data to Elasticsearch.
One benefit is the centralized approach to store data.
This allows to randomly deploy more instances to collect data from other zones.

\begin{figure}
    \centering
    \includegraphics[width=\textwidth]{figures/heicat-architecture.pdf}
    \caption[Visualization of the MADCAT packet flow]{Visualization of the MADCAT packet flow starting at the network interface. The Ethernet and wireless interface forwards packets to the desired module.}
    \label{fig:madcat-architecture}
\end{figure}

\section{Results}

Our MADCAT (28th of October till 17th of November) and T-Pot (16th of November till 6th of December) have been deployed for three weeks respectively.
All instances had a connection to both subnets.
First, the results obtained in the subnet \ipAddress{129.206.218.0/24} will be presented, closing up with the ones we claim in the eduroam network.

In total, MADCAT received \numprint{35372} packets.
Overall, the modules TCP ($66.62\%$) and RAW ($33.26\%$) received the majority of all connection attempts.
The minority with less than one percent are suspicious packets which have individual TCP flags like reset or syn set. 
On contrary, we could not identify any harmful activity based on these packets.
In addition, the T-Pot accounted in total .
Overall, ConPot (), Honeytrap (), and Ciscoasa () received most of the attacks with a total number of $1$.
Interestingly, we could identify \ac{snmp} connections which are used by print servers to discover printers.

\begin{figure}[ht]
    \centering
    \includegraphics[width=\textwidth]{figures/madcat-protocol-usage.pdf}
    \caption[Protocol distribution of MADCAT]{Protocol distribution of MADCAT. ICMP, IPv6 and UDP are the most used protocols. Timestamp; 28th of October to 17th of November.}
    \label{fig:madcat-protocols}
\end{figure}

\autoref{fig:madcat-protocols} shows the protocol distribution indicating a high amount of ICMP and IPv6 packets.
Only $11.59\%$ of all IP address reputations could be resolved, splitting up into known attacker ($11.26\%$), mass scanner ($0.14\%$), bad reputation ($0.12\%$), and tor exit node ($0.08\%$).
Focusing on TCP packets, $88.3\%$ are known attackers with source port 113 as main target.
The port 113 is officially known as the Identification Protocol (IDENT)\cite{rfc1413} used for identification/authorization on a remote server such as POP, IMAP, and SMTP.
Comparing the results with the stateless firewall settings, we spot a potential leak that allows adversaries to send IDENT requests to our network.
Decoding the payload of these TCP packets, rather than compromising systems based on IDENT protocol vulnerabilities attackers used this port to deploy any exploitation attempt possible.
We identified attempts to acquire an SSH session, using SMB and SIP connection attempts, as well as various HTTP requests.
For an example we take a closer look at two payloads that have been sent to our instance.
\autoref{lst:sip-exploitation} outlines an SIP probe that checks if any VoIP service is active by answering the request packet.
Next, \autoref{lst:smb-exploitation} shows an SMB probe trying to achieve the same.
The IP address reputation could help to answer if these packets are sent by a real user or an attacker.
Considering our two examples, both IP addresses were resolved as known attacker, thus, we identified them as a probe packet before executing their attack.
A vital security interest in port 113 is negligible, however, our concept helps to detect such leaks especially when stateless firewalls are the main doorkeeper for packets.

\begin{figure}[ht]
    \centering
    \includegraphics[width=\textwidth]{figures/madcat-overview-map.pdf}
    \caption[Attack distribution of MADCAT]{Attack distribution of MADCAT. USA, Russia, China, and Germany are the most attacking countries. 28th of October to 17th of November.}
    \label{fig:madcat-attack-distribution}
\end{figure}

\autoref{fig:madcat-attack-distribution} shows the attack distribution indicating the origin of an IP address.
Most of the connection attempts are originated from the United States, Germany, and China.
As we have shown beforehand in \autoref{chap:cloud-security}, geographical information only outlines the last know location of a node.
Like our results in heiCLOUD, we assume that this information is not reliable as an indicator from where attacks take place.
Nevertheless, it is an interesting inside to see where the last node originated from.

\begin{figure}[ht]
    \centering
    \includegraphics[width=\textwidth]{figures/madcat-suricata-alerts.pdf}
    \caption[Suricata results of T-Pot]{Suricata results of T-Pot. Timestamp; from 16th of November to 6th of December}
    \label{fig:suricata-distribution}
\end{figure}

Suricata identified odd behaviors in the network (\autoref{fig:suricata-distribution}).
In total, it detected \numprint{300000} alerts and \acsp{cve}.
Besides minor alerts like \textsc{nmap} scans, Suricata registered alerts in \ac{snmp} requests, TCP stack, and Wind River VxWorks.
CVE-2020-11899 \cite{CVE-2020-11899} accounts nearly $99\%$ with a total number of $\numprint{102,838}$.
This CVE is one of 19 others forming the \enquote{Ripple20} vulnerability in the low-level TCP/IP library developed by Treck, Inc..
One of the task of the Track TCP/IP stack is to reassemble fragmented packets.
Whenever a fragmented packet arrives the stack tries to validate to total length in the IP header.
If the total length is not correct, it trims the data.
However, this leads to an inconsistency, and thus, resulting in a buffer overflow when someone sends fragmented packets through tunnel.
A detailed description of the vulnerability can be found in \cite{ripple20}.
Adversary could send malformed IPv6 packets that cause an Out-of-bounds Read which results in a potential remote code execution.
Only TCP/IP stack versions until $6.0.1.66$ are affected by this vulnerability.
Nevertheless, the tremendous amount of alerts shows the importance of adapting the IPv6 permits.
Second most recorded vulnerability with the highest score is CVE-2002-0013 \cite{CVE-2002-0013} that allows remote attackers to cause a denial of service or gain privileges in the SNMPv1 protocol.
The root cause for the CVE alert is the usage of the default public community for broadcast requests instead of configuring a private community with mandatory authentication.
To compromise SNMP, attackers have to have access to the protocol.
However, the University firewall blocks SNMP port 161 and 162 for TCP and UDP, thus, restricting any access from outside.
If adversaries plan to deploy an attack on the SNMP protocol, they need to have a connection to the internal network.
Acquiring a connection to the internal network at University seems to be rather hard to accomplish.
On the contrary, all connection attempts registered by our concept have been made within network, and they do reflect a normal SNMP communication.
Lastly, Wind River VxWorks 6.9.4 and vx7 in CVE-2019-12263 \cite{CVE-2019-12263} cause a Buffer Overflow due to the underlying TCP component that results in a race condition.
Each connection attempt with CVE-2019-12263 originated from Russia.
Hence, we assume that the source IP address had a vicious intention to send an urgent flag.
However, for the other CVE's the IP reputation could not be resolved.

\begin{figure}
    \lstinputlisting[language=bash, caption={[MADCAT connection attempt to exploit SIP connection]MADCAT connection attempt to exploit SIP connection. Received on the 16th of November. IP reputation: known attacker. Location Germany.}, label={lst:sip-exploitation}]{listings/exploit-sip.txt}
\end{figure}

\begin{figure}[ht]
    \centering
    \includegraphics[width=\textwidth]{figures/madcat-port-histogram.pdf}
    \caption[Attack port histogram]{Attack port histogram. Timestamp; from 16th of November to 6th of December}
    \label{fig:port-histogram}
\end{figure}

Results from our T-Pot instance are exiguous, and in short, no real attacks such as shell exploitation have been performed.
Overall, all connection attempts originated from Germany within the same network, and are made on port $161$ and $4567$.
Conpot registered minor SNMPv2 Get, SNMPv1 Get, and GetNext requests.
A possible attack vector could be an SNMP reflection/amplification attack.
As previously discussed our assumption is that devices within the network have a misconfigured printer and send broadcast requests frequently to find the machines.
These SNMP requests affiliate with day-to-day traffic in an internal network, and thus, are not suspicious.
Second most attacked honeypot is Honeytrap which received numerous packets on different ports whereas $39\%$ evince an empty payload.
All of these received packets have a resolved IP address in the subnet \ipAddress{129.206.0.0/16}.
It remains unclear if these connections are malicious or are acquired by accident.
Investigating the payload of outliers does not confirm the assumption of a vicious intention.
Thus, declaring these results as negligible.
Overall, most of the connection attempt received by our instance are from these IP addresses: \ipAddress{129.206.217.118}, \ipAddress{129.206.218.23}, and \ipAddress{129.206.218.194}.

\begin{figure}
    \lstinputlisting[language=bash, caption={[MADCAT connection attempt to exploit SMB connection]MADCAT connection attempt to exploit SMB connection. Received on the 16th of November. IP reputation: known attacker. Location Germany.}, label={lst:smb-exploitation}]{listings/exploit-smb.txt}
\end{figure}

Lastly, we consider the results from the eduroam network.
Neither T-Pot nor MADCAT could identify any significant behavior over a period of three weeks.
Unlike the subnet \ipAddress{129.206.218.0/24}, our honeypot did not register any suspicious packets, TCP flags, or other \acp{cve}.
In retrospect, the eduroam configuration has shown to work as designed.
Thus, our client seemed to be encapsulated from others, and receives no other packets.

Besides the subtle output we have received, our results have given an inside of the value of honeypots in a restricted network zone.
For the Heidelberg University, using honeypots to evaluate their stateless firewall have never been considered.
We have shown that our initial concept delivered minor findings in the subnet \ipAddress{129.206.218.0/24} with stage 1 firewall.
As a result the port 113 used for the IDENT protocol will be removed in the future to reduce the attack surface, thus, contributing to the firewall definition.
Overall, our two instances received numerous packets containing interesting payloads.
In comparison to our T-Pot that has been deployed in heiCLOUD, our results are as expected delicate and the data analyzation results in a more detailed manner.
The statement from \citet{Spitzner2003} that honeypots only receive little input and nearly ever input is suspicious matches our results only half way.
As shown beforehand, our results are dramatically little, however, only a few requests seemed suspicious.
Nonetheless, we could answer our initial question if attackers have any access to the restricted network zone at the Heidelberg University.

\section{Discussion}

We have seen that honeypots do help to find potential leaks in restricted network zones.
However, it remains questionable if our concept is capable of delivering correct results.
Our instance has been running for three weeks in the subnet \ipAddress{129.206.5.0/24}.
However, to deliver any meaningful entries our honeypot needs to be detected as vulnerable target.
We could not detect any large scans on our instance, thus, it is very likely that either an attacker could not find our instance or no one had any kind of access.
